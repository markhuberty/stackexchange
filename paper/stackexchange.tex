%\documentclass[14pt]{extarticle}
\documentclass[11pt]{article}
\usepackage{natbib}
\usepackage[dvipdfm,colorlinks=true,urlcolor=DarkBlue,linkcolor=DarkBlue,bookmarks=false,citecolor=DarkBlue]{hyperref}

\usepackage[pdftex]{graphicx}
\usepackage{fancyhdr}
\usepackage[T1]{fontenc}
\usepackage{palatino}
\usepackage[utf8]{inputenc}
%\usepackage[super]{nth}
\usepackage{setspace}
\usepackage{placeins}
\usepackage{subfigure}
\usepackage{multirow}
\usepackage{rotating}
\usepackage{marvosym}  % Used for euro symbols with \EUR
\newcommand{\HRule}{\rule{\linewidth}{0.5mm}}
\usepackage{longtable} %% Allows the use of the longtable format produced by xl2latex.rb
\usepackage{lscape} %% Allows landscape orientation of tables
\usepackage{appendix} %% Allows customization of the appendix properties
\setcounter{tocdepth}{1} %% Restricts the table of contents to the section header level entries only

\usepackage{geometry}
\geometry{letterpaper}
\usepackage{amsmath}
\usepackage[stable]{footmisc}

%% The following settings are for the listings environment in R
\usepackage{listings}
\usepackage[svgnames]{xcolor}
\usepackage{soul}
\sethlcolor{LightGoldenrodYellow}
\lstset{backgroundcolor=\color{LightYellow}}
\lstset{framextopmargin=6pt, framexbottommargin=6pt, framerule=4pt, rulecolor=\color{White}}

\title{Individual-level evidence of IT skill specialization and formation
  in the advanced industrial economies\thanks{DRAFT NOT FOR DISTRIBUTION}}
\author{Mark Huberty\thanks{Travers Department of Political Science,
    University of California, Berkeley. Contact:
    \url{markhuberty@berkeley.edu}.}}
\date{\today}

\graphicspath{{../figures/}}

\begin{document}

\maketitle
\doublespacing
\begin{abstract}
  Typologies of capitalism emphasize the role of skill formation as
  both a shaper of and a response to firm incentives for
  specialization in different forms of technology. However, actual
  worker skill is usually difficult to measure and costly to
  observe. We present novel estimates of IT worker skill based on
  actual observation of skills in practice. We exploit a new data
  source that allows observation of tens of thousands of users across
  over fifty countries and three years. Based on these data, and
  subject to several caveats about the worker population, we find (). 
\end{abstract}
\section{Introduction}
\label{sec:introduction}

Theories of comparative political economy have put substantial
emphasis on the interaction of welfare state incentives, worker skill,
and firm choices about how and where to specialize. However,
measurements of worker skill are often quite coarse. The difficulty of observing individuals at
work, under similar conditions, across countries has led researchers
to rely on aggregate
statistics such as the level of vocational training, the share of
university-educated individuals in the workforce, and the presence of
apprenticeship programs. 

We present novel measurements of highly granular skill in a large
information technology (IT) worker population across many countries. Using data from an open,
online IT question-and-answer forum, we are able to specify exact
definitions for skill, skill diversity, and expertise. Using those
definitions, we can construct very detailed comparisons of comparative
IT worker skill specialization across both advanced industrial and
emerging market economies. We believe this is the first use of this
dataset to measure worker skill.

Within the resolution of the data, and subject to several caveats on
the selection process, we find ().



\section{Institutions and skill development}
\label{sec:inst-skill-devel}


Theories of comparative political economy predict that institutional
variation in the advanced industrial economies will create diverging
patterns of skill formation in workers, and in technology adoption and
exploitation among firms. The Varieties of Capitalism literature in
particular \citep{Hall:2001} makes two predictions about technology
and skills: first, that workers in more liberal economies
will adopt more general skills than their counterparts in coordinated
economies; and second, that firms in liberal economies will be more
likely to engage in radical innovation and adopt leading-edge
technology. 

These claims rest on an implicit microeconomic model of firm and
worker behavior. In more liberal economies, the lack of employment
protection laws means that workers must consider the possibility of
losing their jobs without recourse. Likewise, firms must consider that
workers who have no protection will feel little loyalty to any given
firm. In that context, both workers and firms will converge on an
equilibrium that values general skills--the workers, so that they can
easily find jobs at other firms, and the firms, so that workers are
easily replaced. Conversely, countries with strong employment laws
will tend to favor specific skills, because firms must extract high
value from employees that cannot be easily dismissed, and because
employees believe that investment in firm-specific skills will be
rewarded by long employment tenure. 

Empricially, the evidence used to support these claims has typically
relied on macro-level measures of skill formation, such as the share
of workers in vocational or continuing education (taken to be
representative of firm-specific skills), the share of
university-educated workers (taken to represent more general skills),
and the role of apprenticeship programs in skill formation.\footnote{For a full set of metrics, see
  \cite{hall2009varieties}.} But these metrics do not directly measure
worker-level behavior. Moreover, they may mis-identify skill
specificity. For instance, we may believe that a manager with a
liberal arts degree from a university has cultivated very general
skills. But a university-educated statistician working in natural
language processing in Silicon Valley has a highly specific skillset
useful for a relatively small set of highly specialized firms. Whether
that individual should be lumped into the same category as a manager,
or instead treated like a highly skilled machinist graduate of an
\textit{Azubi} program in Baden-Wurtemburg is unclear. 

\section{Pursuing new data for measurement}
\label{sec:data}

I exploit a new dataset on technology adoption and skill formation in
the information technology industry. That dataset provides
individual-level evidence of cross-sectional and intertemporal
patterns of skill formation among workers worldwide, and augments that
data with measures of expertise and evidence of patterns of worker
interaction. 

StackExchange represents a collection of community-created and
maintained internet sites that field questions and answers on topics related to information
technology. Begun in 2008 by Joel Spolsky and colleagues, it has grown
into one of the most comprehensive communities of technical
information in the world. As of 2011, it reported approximately
725,000 users, 1.2 million site visits per day, and 4.2 million separate
answers covering 82\%  of the 1.9 million submitted questions. It had grown from its initial site,
StackOverflow, to sites covering everything
from programming to data security to cooking and philosophy. Anecdotal
evidence suggested that the quality of the site had led recuiters to
ask for samples of Stackoverflow questions and answers from potential
hires, as one measure of worker skill.\footnote{cite}

The format of StackExchange interactions can be stylized as follows: a
user poses a question and tags the question with metadata related to
the specific technologies or technological domains the question
pertains to. Other users respond to the question. Those responses, in
turn, are rated by respondents, the person who posed the original
question, and other community members, on the basis of accuracy,
solution elegance, and completeness. Questions and answers can be
retrieved by querying metadata, the user ID of the questioner or
answerer, and other boolean search terms. 

\begin{figure}[ht]
  \centering
  \includegraphics[width=\textwidth]{stack_question_answer}
  \caption{Sample StackOverflow question and answer}
  \label{fig:stack-question-answer}
\end{figure}

Users contribute to StackExchange through user accounts that contain a
range of information about the user. Users may, but are not required
to, provide information about their geographic location, interests,
website address, and other personal characteristics. There is no
evidence that this data is validated, and avatars or similarly
obscured or stylized user names and identities are permitted. 

%% Things to think about here: answered vs. unanswered questions;
%% multiple tags, community-modded responses, etc.

All StackExchange data is collected and published under a Creative
Commons license. In keeping with the license, StackExchange releases
a bi-monthly dump of its entire dataset, including the full record of
questions and answers, user profiles, record tagging and metadata, and
time stamps. The data are anonymized with respect to user names but
not other user metadata, including geographic location, web site,
the questions posed or answered by each user, and the
user's community-determined ``reputation''.\footnote{For more
  information on the StackExchange data dump, see
  \url{http://blog.stackoverflow.com/category/cc-wiki-dump/}. 
  %For information on how the data are rendered anonymous, see
  %\url{}. 
   For information
  on the terms of use for the Creative Commons license, see
  \url{http://creativecommons.org/licenses/}. For academic use of the
  data, including studies of social interaction and exchange markets,
  see \url{http://blog.stackoverflow.com/2010/05/academic-papers-using-stack-overflow-data/}.}

These data therefore represent a rich trove of information about the
interaction of IT workers worldwide. The dataset contains not only the location
and other information about the community members themselves, but also,
through the information on questions posed and answered and their reputation, the
kinds of technology they work with and are expert in, and their rating by
a community of their peers. As such, the data provide an opportunity
to directly observe patterns of specialization of knowledge and
interaction among IT workers across both countries and time. 

\section{Research Design}
\label{sec:research-design}

To employ the StackOverflow data in the study of skill and task
specificity, we first need to define exactly what is meant by the
terms task and skill, and how specificity will be defined. 

\subsection{Definitions}
\label{sec:definitions}

\subsubsection{Skill}
\label{sec:skill-specificity}

\textbf{Skills} will refer to knowledge of the workings of specific
technologies or technological concepts. These may include, for
instance, language-specific knowledge in $C$, knowledge of algorithmic
implementations of specific programming or computer science concepts,
or operational knowledge of how to set up and administer specific
kinds of software or software systems. These skills may, in fact, be
independent from one another. Just as someone might be very knowledgeable about fuel
injection systems, but not necessarily ever need to rebuild them, so
could an IT worker be very knowledgeable about the $C$ programming
language without every using it to, say, optimize highly parallelized
programs used for scientific computing.  

We can define skill on the basis of the metadata attached to each
question. We treat each unique metadata tag as a member of a skill
domain. Skill domains in turn are constructed from patterns of
co-occurrance of metadata tags in questions. Formally, we construct
the proximity matrix $P$ from a set of questions $Q$, each of which
contains a subset of tags $t_Q \in T$. We can construct the proximity
of tag $t_i$ to tag $t_j$ as the conditional probability that a
question containing $t_i$ also contains $t_j$. In cases where $P(t_i,
t_j) \neq P(t_j, t_i)$, we take the maximum conditional
probability.\footnote{The intuition here is straightforward. Consider
  the \texttt{bash} shell environment for unix. $P(bash | unix)
  \approx 1$, since \texttt{bash} is almost exclusively a unix
  tool. But $P(unix | bash) \approx 0$, since there are many, many
  unix utilities. Taking the minimum conditional probability would
  wildly understate the relationship between \texttt{bash} and the
  Unix operating system ecosystem.}

\subsubsection{IT workers}
\label{sec:it-workers}

We define an IT worker as a contributor to and participant in the
question-and-answer process on Stackoverflow. The April 2011 data dump
provides information on approximately 700,000 individual users. Within
this group, we identify two separate
groups of users: those who provide geographic information in their
user profiles (approximately 130,000 individuals) and those who do
not. For those users who provide geography information, we
disambiguate that data using the Yahoo GeoDict webservice.\footnote{\url{http://developer.yahoo.com/geo/placemaker/}} This
service provides a scriptable API which translates unstructured
geographic information into a standardized format.\footnote{For
  instance, variants like ``Karlsruhe'', ``Karlsruhe, Germany'', and
  ``Karlsruhe, Deutschland'' are all translated into a common format
  specifying city and country in a parsable JSON format.}

\subsubsection{IT worker skill}
\label{sec:it-worker-skill}

Each worker $w \in W$ is associated with a set of answers $A_w$. Each
answer is associated to a question $Q$, which has a set of metadata
tags $T_Q$. We assign each question to a skill or skills based on the
share of tags found in skill cluster $S$. Thus, for instance, a
question that had four tags, for which two were found in one skill
cluster and two in another, would be scored as 50\% skill 1 and 50\%
skill 2.  

User skill is thus defined as the quality of answers they provide to
questions associated with specific skills. Each answer has a set of
votes $V_A \in [0, \infty)$. We assign skill specialization to users
as the skill-share-weighted sum of votes on answers to
questions. Hence if a user answered the question described above (with
a 50/50 skill split), their skill score for that question would be the
total vote count for that answer, spilt 50/50 between two skill
categories. Based on all user answers, then, we can aggregate a total
skill score as the sum of skill scores for all questions the user has
answered.  Formally, then, skill is defined as a vector $s_i \in S_U$:

\begin{equation}
  \label{eq:4}
  s_i = \sum_{A_w, s} V_{A_w} * \alpha_{Q, s}
\end{equation}

Where $\alpha_{Q,s}$ is the share of question $Q$ dealing with skill
cluster $s$.

\subsubsection{IT worker skill specialization}
\label{sec:it-worker-skill-1}


\section{Characterizing the selection problem}
\label{sec:char-select-probl}

We have no reason to believe that the Stackoverflow user population is
particularly representative of the IT worker population at large. The
question, then is, twofold:
\begin{enumerate}
\item Does the selection problem inhibit internal validity (comparison
  among user groups within the user population)?
\item How might users differ from non-users?
\end{enumerate}

We first provide comparative statics on the user population in general
and across countries. 

We first question whether linguistic ability shapes
participation. Stackoverflow is an English-language website, which may
limit participation from non-English-speaking populations. Conversely,
IT is a largely English-dominated sector for historical reasons, and
IT workers are generally highly educated. Figure \ref{} shows that the Anglo-Saxon and
Scandinavian countries participate at higher per-capita rates than
Continental Europe or other countries. However, figure \ref{} also
shows that aggregate user reputatation--a measure of aggregate
skill--does not appear to correlate with aggregate measures of
English-speaking ability once we control for when users joined the
user community. Hence while English ability (higher in Anglo-Saxon
countries and Scandinavia) may affect selection into the user
population, it does not appear to affect patterns of participation
once in the user community. 

We second ask whether we observe different rates of adoption among
countries. Figure \ref{} provides some evidence that the early
adopters are almost entirely English-speaking. But while the timing of
adoption changes, we find evidence that patterns of adoption are quite
similar. Figure \ref{} shows that, when we normalize the timing of
adoption such that day 0 for all countries is the first day of user
adoption for that country, adoption patterns follow a generally
sigmoidal pattern with a similar curve shape for all countries and
country groups. Moreover, the Scandinavian countries again are highly
similar to the English-speaking economies, while the Continental
European economies maintain lower, adoption rates but similar adoption
patters at those rates.

Third, we ask whether per-capita rates of participation differ
substantially across countries. We only observe user skill to the
extent that they contribute answers to questions posed by
others. Substantial differences in per-capita participation rates
may lead to observed skill differences that do not accurately reflect
actual skill differences, but instead reflect participation rates
among the user population. Figure \ref{} provides some evidence that
the per-capita rate of answer contribution across countries is highly
similar. 


% This proximity matrix,
% of dimension $T$, can be treated as a graph of nodes $T$ and edges
% $t_i, t_j$. Based on this graph, we can construct skill families from
% the clustering of nodes in the graph. We employ the MCL community
% detection algorithm \citep{dongen2000} which operates very quickly on
% large graphs with many verticies (28,000 in this case) and
% edges. Inspection of the identified clusters indicates a high degree
% of internal coherence. Appendix \ref{sec:mcl-cluster-examples}
% provides several examples of the resulting clusters of tags. 

% We may infer skill specificity from a question-and-answer site
% like StackOverflow by analyzing the co-occurrance of technologies by
% question. If questions about some technology often reference one or
% many other technologies, we may infer that this technology is a more
% \textit{general} technology than one for which questions ask about it
% alone. Thus, for instance, the $C$ programming language is a
% general-purpose technology used for everything from commercial
% software development to scientific computing. In contrast, the
% \textsc{scada} language is used predominately for control of
% industrial machinery. 

% To quantify this definition of specificity, we must first define the
% relationship of technologies to each other. We can quantify this
% relationship as the proximity between any two technologies in the
% question space. Each question
% $q_i, i \in [1...n]$ considers some technologies $T_q$, which are a subset of the entire set of technologies $T_Q$ considered in all
% questions $Q$. The proximity between any two technologies $T_i$ and
% $T_j$ $\in T_Q$ can be defined as the conditional
% probability that they appear in questions together. Formally, $P(T_i
% \in T_q | T_j \in T_q)$ can be defined as:

% \begin{equation}
%   \label{eq:1}
%   p_{i,j} = \frac{\sum_{q \in Q} T_i \in T_q | T_j \in T_q}{\sum_{q \in Q} T_j \in T_q}
% \end{equation}



% \subsubsection{Task proximity}
% \label{sec:task-specificity}

% \textbf{Tasks} will refer to the application of
% technologies. Consistent with the definition in equation \ref{eq:1},
% we will begin to define task specificity by first defining task
% proximity. Task proximity is the conditional probability of
% co-occurrance of tasks in the question metadata. Thus, for instance, if questions regarding thread
% optimization also deal with memory optimization, but rarely deal with
% web page design, then we would regard thread and memory optimization
% as more proximate tasks than thread optimization and web page design. Using the same notation as above, 

% Each question
% $q_i, i \in [1...n]$ considers some tasks $K_q$, which are a subset of the entire set of tasks $K_Q$ considered in all
% questions $Q$. The proximity between any two tasks $K_i$ and
% $K_j$ $\in K_Q$ can be defined as the conditional
% probability that they appear in questions together. Formally, $P(K_i
% \in K_q | K_j \in K_q)$ can be defined as:

% \begin{equation}
%   \label{eq:2}
%   \frac{\sum_{q \in Q} K_i \in K_q | K_j \in K_q}{\sum_{q \in Q} K_j \in K_q}
% \end{equation}

% \subsubsection{From proximity to specificity}
% \label{sec:from-prox-spec}

% Scoring tasks or skills as more or less specific requires that we
% translate from proximity between pairs of tasks or skills to a overall
% measure of specificity. Given a proximity matrix $P$ that contains the
% pairwise proximities $p_{i,j}$ between all technologies (skills) $T_i, T_j \in T_Q$, we may define the
% specificity of any technology (skill) $T_i$ as in equation \ref{eq:3}, where
% $\delta \in (0,1]$ is some threshold proximity value.

% \begin{equation}
%   \label{eq:3}
%   ST_i = \sum_{q \in Q} p_{i,q} > \delta 
% \end{equation}

% Higher-valued $ST_i$ indicate skills that are more general, defined as
% applied in concert with a larger set of other skills. We can write the
% same relationship for task specificity $SK_i$.

% \subsubsection{Task specificity and technological clustering}
% \label{sec:task-spec-techn}

% Observers will note that some technologies (skills) are
% heirarchical. For instance, the \texttt{bash} shell scripting
% environment is a subset of the Unix operating system. Thus $P(unix |
% bash)$ will be close to 1. In aggregate, leaving this fact unaccounted
% for would simply reproduce heirarchies like this. To account for this
% problem, we take the minimum of the pairwise conditional probabilities
% for each skill or task pair. Thus while $P(Unix, bash)$ may be 1,
% $P(bash, Unix)$ will most likely be far smaller, reflecting the fact
% that only a portion of questions about Unix will inquire about the
% \texttt{bash} environment.

% \textbf{Alternatively}, we can view this heirarchy of technologies as
% a natural clustering of technology types. Bridges between the
% technology types consist of nodes describing tasks undertaken in
% either technology. Thus, for instance, we may
% observe a cluster of ``web technologies'' that are separate from
% ``operating systems'' and ``machine control''. They may each be joined
% by a common node ``memory management''. Viewed in this fashion,
% users adopt skills in a technology cluster, rather than a set of atomistic
% technologies. A user's skill specificity is then measured as the number of
% clusters that he or she specializes in. 

\subsubsection{Individual-level skill and task specialization}
\label{sec:indiv-level-skill}

Given these definitions of task and skill specificity, we can now
define what we mean by those terms for any given individual. for a
user $u \in U$, their skill specificity is the mean specificity in all
questions $q \in Q$ that they have answered. Task specificity is
defined the same way. Each user thus receives both a skill and a task
specificity score. Conceptually, then, each user falls into one of
four possible categories: tackling specific tasks with general skills,
general tasks with general skills, specific tasks with specific
skills, or specific tasks with general skills. Notice, here, however,
that we are not bound by discrete categorization. Rather, we can
measure both task and skill specialization as continuous quantities.

Using the alternative definition of task proximity, we can define
specialization as the number of different technology clusters in which
a given user displays expertise. This eliminates the (potentially
artificial) distinction between tasks and skills.

\subsection{Technical design}
\label{sec:technical-design}

\subsubsection{Option 1: continuous measures of tasks and skills}
\label{sec:option-1:-continuous}

\begin{enumerate}
\item For the full set of users $U$, select those with geo-coded
  information in their profiles, $U_g$ (DONE)
\item For $U_g$, disambiguate the geo-coded data (DONE, via the Yahoo
  Placemaker service)
\item For all questions $Q$, get two subsets of questions:
  \begin{enumerate}
  \item All questions for which the submitter and AT LEAST one
    answerer are in $U_g$
  \item All questions for which AT LEAST one answerer are in $U_g$
  \end{enumerate}
\item Get all tags $T$ in the question set $Q$. Subset for only those
  tags $T^*$ that occur more than $N$ times in the dataset. Construct a
  proximity matrix $M_{T^*}$ with the conditional probability of tag
  co-occurrance as defined in equation \ref{eq:1}
\item For all tags $T^*$, categorize as either skills or tasks
\item For all users $U_g$, get all tags in all questions answered by
  $U_g$ and the community-generated score of the answer. Compute the
  weighted task and skill specificity measures as described in
  equation \ref{eq:3} and section \ref{sec:indiv-level-skill}
\end{enumerate}

\subsubsection{Option 2: clusters of tags}
\label{sec:option-2:-clusters}

\begin{enumerate}
\item Get the users and tags as described in section \ref{sec:option-1:-continuous}.
\item For tags $T$, construct the graph $G_t$ of tags wherein each tag is a
  node and the edges are defined as the conditional probability of
  co-occurrance. Only use edges for which $p > threshold$.
\item For $G_t$, find the graph k-cores (see
  \url{http://networkx.lanl.gov/reference/algorithms.core.html} and
  associated references).
\item For each k-core, get the tags associated with it
\item For each user $U_g$, determine which tags $T' \in T^*$ apply to
  their answers
\item Count the number of tags they answer, in the number of k-cores,
  for each user in $U_g$
\end{enumerate}

\section{Hypotheses}
\label{sec:hypotheses}

\subsection{Approach 1: continuous measures}
\label{sec:appr-1:-cont}

\begin{description}
\item[H1] Users from CME countries will have higher mean specificity,
  but in fewer skills or tasks (i.e., will display strong but
  relatively narrow expertise)
\item[H2] Users from LME countries will have lower mean specificity
  scores, but will display expertise in more skills or tasks (i.e.,
  will display weaker but broader expertise)
\end{description}

\subsection{Approach 2: clustered measures}
\label{sec:appr-2:-clust}

\begin{description}
\item[H1] Users from LME countries will display expertise in more
  technology ``clusters'' than users from CME countries. This is
  equivalent to saying that LME users will have a more general skill
  set, as a function of more liquid labor markets.
\item[H2] Users from CME countries will display expertise in more tags
  in any given cluster than users from LME countries, even if they are
  present in fewer technology clusters
\end{description}



\section{Results}
\label{sec:results}

\section{Discussion}
\label{sec:discussion}

\section{Conclusions}
\label{sec:conclusions}

\appendix
\appendixpage

\section{MCL cluster examples}
\label{sec:mcl-cluster-examples}



\FloatBarrier
\bibliography{/bibs/VOC_Bib}
\bibliographystyle{apalike}
\end{document}