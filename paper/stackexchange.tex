%\documentclass[14pt]{extarticle}
\documentclass[11pt]{article}
\usepackage{natbib}
\usepackage[dvipdfm,colorlinks=true,urlcolor=DarkBlue,linkcolor=DarkBlue,bookmarks=false,citecolor=DarkBlue]{hyperref}

\usepackage[pdftex]{graphicx}
\usepackage{fancyhdr}
\usepackage[T1]{fontenc}
\usepackage{palatino}
\usepackage[utf8]{inputenc}
%\usepackage[super]{nth}
\usepackage{setspace}
\usepackage{placeins}
\usepackage{subfigure}
\usepackage{multirow}
\usepackage{rotating}
\usepackage{marvosym}  % Used for euro symbols with \EUR
\newcommand{\HRule}{\rule{\linewidth}{0.5mm}}
\usepackage{longtable} %% Allows the use of the longtable format produced by xl2latex.rb
\usepackage{lscape} %% Allows landscape orientation of tables
\usepackage{appendix} %% Allows customization of the appendix properties
\setcounter{tocdepth}{1} %% Restricts the table of contents to the section header level entries only

\usepackage{geometry}
\geometry{letterpaper}
\usepackage{amsmath}
\usepackage[stable]{footmisc}

%% The following settings are for the listings environment in R
\usepackage{listings}
\usepackage[svgnames]{xcolor}
\usepackage{soul}
\sethlcolor{LightGoldenrodYellow}
\lstset{backgroundcolor=\color{LightYellow}}
\lstset{framextopmargin=6pt, framexbottommargin=6pt, framerule=4pt, rulecolor=\color{White}}

\title{Individual-level evidence of IT skill specialization and formation
  in the advanced industrial economies\thanks{}}
\author{Mark Huberty\thanks{Travers Department of Political Science,
    University of California, Berkeley. Contact:
    \url{markhuberty@berkeley.edu}.}}
\date{\today}

\graphicspath{{../figures/}

\begin{document}

\maketitle
\doublespacing

\section{Introduction}
\label{sec:introduction}
 
I present new evidence of comparative patterns of specialization and
technology adoption in the advanced industrial economies. 

\section{Institutions and skill development}
\label{sec:inst-skill-devel}


Theories of comparative political economy predict that institutional
variation in the advanced industrial economy will create diverging
patterns of skill formation in workers, and in technology adoption and
exploitation among firms. The Varieties of Capitalism literature in
particular \citep{Hall:2001} makes two predictions about technology
and skills: first, that workers in more liberal economies
will adopt more general skills than their counterparts in coordinated
economies; and second, that firms in liberal economies will be more
likely to engage in radical innovation and adopt leading-edge
technology. 

Empricially, the evidence used to support these claims has typically
relied on macro-level measures of skill formation, including
rates university versus vocational training, patenting rates across
different kinds of technology, and aggregate R\&D
spending.\footnote{For a full set of metrics, see
  \cite{hall2009varieties}.} But these metrics do not directly measure
worker-level behavior.

\section{Data}
\label{sec:data}

I exploit a new dataset on technology adoption and skill formation in
the information technology industry. That dataset provides
individual-level evidence of cross-sectional and intertemporal
patterns of skill formation among workers worldwide, and augments that
data with measures of expertise and evidence of patterns of worker
interaction. 

StackExchange represents a collection of community-created and
maintained internet sites that field questions and answers on topics related to information
technology. Begun in 2008 by Joel Spolsky and colleagues, it has grown
into one of the most comprehensive communities of technical
information in the world. As of 2011, it reported over () questions,
() users, and () daily page views. It had grown from its initial site,
StackOverflow\footnote{In programming parlance, a stack overflow
  refers to a program condition in which memory usage exceeds
  memory allocation, causing a program crash. Hence StackOverflow is,
  in a sense, excess memory for humans.}, to sites covering everything
from programming to data security to cooking and philosophy. 

The format of StackExchange interactions can be stylized as follows: a
user poses a question and tags the question with metadata related to
the specific technologies or technological domains the question
pertains to. Other users respond to the question. Those responses, in
turn, are rated by respondents, the person who posed the original
question, and other community members, on the basis of accuracy,
solution elegance, and completeness. Questions and answers can be
retrieved by querying metadata, the user ID of the questioner or
answerer, and other boolean search terms. 

Users contribute to StackExchange through user accounts that contain a
range of information about the user. Users may, but are not required
to, provide information about their geographic location, interests,
website address, and other personal characteristics. There is no
evidence that this data is validated, and avatars or similarly
obscured or stylized user names and identities are permitted. 

%% Things to think about here: answered vs. unanswered questions;
%% multiple tags, community-modded responses, etc.

All StackExchange data is collected and published under a Creative
Commons license. In keeping with the license, StackExchange releases
a bi-monthly dump of its entire dataset, including the full record of
questions and answers, user profiles, record tagging and metadata, and
time stamps. The data are anonymized with respect to user names but
not other user metadata, including geographic location, web site,
the questions posed or answered by each user, and the
user's community-determined ``reputation''.\footnote{For more
  information on the StackExchange data dump, see \url{}. For information
  on how the data are rendered anonymous, see \url{}. For information
  on the terms of use for the Creative Commons license, see \url{}.}

These data therefore represent a rich trove of information about the
interaction of IT workers worldwide. The dataset contains not only the location
and other information about the community members themselves, but also,
through the information on questions posed and answered and their reputation, the
kinds of technology they work with and are expert in, and their rating by
a community of their peers. 

 


\section{Research Design}
\label{sec:research-design}

\section{Results}
\label{sec:results}

\section{Discussion}
\label{sec:discussion}

\section{Conclusions}
\label{sec:conclusions}


\FloatBarrier
\bibliography{/bibs/VOC_Bib}
\bibliographystyle{apalike}
\end{document}